\documentclass[twoside,11pt]{article}

\usepackage{style/jmlr2e}

\begin{document}

\title{Summative Assignment\\ 
Artificial Intelligence COMP2261 –\\
Machine Learning 2020/2021}

\author{\name Matthew Chapman \email matthew.chapman@durham.ac.uk \\
       \addr Department of Computer Science\\
       Durham University\\
       Durham, United Kingdom}

\maketitle

\begin{abstract}%  
This paper describes
\end{abstract}

\begin{keywords}
  Bayesian Networks,
\end{keywords}

\cite{kramer1991nonlinear}

\section{Problem framing (10\%)}
The problem I plan to solve is of predicting the number of Covid-19 cases in the future, which can be tomorrow or next week. 

The motivation is to create machine learning models trained on real data aid the analysis of the pandemic and inform public health decision making.   

\section{Experimental procedure (35\%)}
\begin{itemize}
    \item Clean the dataset.
    \item Split the dataset into training and test sets.
    \item Train a logistic regression model.
    \item Train a polynomial regression model.
    \item Train a normal regression model.
\end{itemize}

\section{Results (25\%)}
\begin{itemize}
    \item Make comparisons between the 3 predictive models
    \item Provide necessary tables and charts to summarise and support the comparisons.
\end{itemize}

\section{Discussions (20\%)}
\subsection{Chosen models}
\subsection{Experimental procedure}
\subsection{Limitations}

\section{Conclusions and lessons learnt (10\%)}
\begin{itemize}
    \item Discuss the results and draw conclusions from your experimentation
\end{itemize}




\vskip 0.2in
\bibliography{references}

\end{document}